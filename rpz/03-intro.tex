\section*{ВВЕДЕНИЕ}
\addcontentsline{toc}{section}{ВВЕДЕНИЕ}
	В настоящее время объем доступной геномной информации продолжает расти. Возникает вопрос о том, как интерпретировать ее для выделения кодируемых черт и поведения. Сравнительная геномика -- это область, которая ставит своей целью получение новых знаний о наследуемых признаках и эволюционных маркерах с помощью сравнительного анализа последовательностей\cite{1_hardison2003comparative}. В частности, сравнительная геномика ставит задачу обнаружения гомологичных участков. Понятие гомологичности может быть определено как наличие общего происхождения\cite{1_hardison2003comparative} у организмов-носителей, обычно выводимого из избыточной схожести строк\cite{2_pearson2013introduction}. Предполагая, что общие черты закодированы в подпоследовательности ДНК передаваемой между огранизмами-носителями, поиск гомологичных участков позволяет рассчитать филогенетическое расстояние\cite{1_hardison2003comparative,2_pearson2013introduction} между организмами-носителями и определить функциональные участки ДНК\cite{1_hardison2003comparative,2_pearson2013introduction}.
	
	Цель данной работы -- разработка метода повышения скорости сравнения геномов с использованием графовой базы данных. В рамках поставленной цели необходимо решить следующие задачи:
	\begin{enumerate}
		\item анализ и классификация существующих методов решения;
		\item формализациия графовой модели данных для представления геномной информации;
		\item разработка алгоритма для сравнения геномов с использованием формализованной модели;
		\item реализация хранилища данных согласно формализованной модели, а также разработанного алгоритма сравнения геномов.
	\end{enumerate}
\pagebreak